% Document class options:
% 11pt for 11 point font
% oneside for not weird margins between odd and even pages
% (omitting oneside gives a better margin setup for print
% but i'm guessing  you're handing this in)
% a4paper to A4 page size, because LaTeX by default is setup
% to use US Letter paper size, because Donald Knuth
% the article class, as the book class is designed for
% larger documents and you probably don't need chapter
% environments
\documentclass[11pt, oneside, a4paper]{article}
\usepackage[utf8]{inputenc}
\usepackage[english]{babel}

\usepackage{hyperref}
\hypersetup{
    colorlinks=true,
    linkcolor=black,
    filecolor=magenta,      
    urlcolor=cyan,
}

\urlstyle{same}

\usepackage[utf8]{inputenc}
\usepackage{graphicx}
\usepackage{array}


% Set spacing (i set it to 1.2x)
\renewcommand{\baselinestretch}{1}
% Indentation (set this to zero for normal prose)
\setlength{\parindent}{0em}
% Line breaking (spacing between paragraphs)
\setlength{\parskip}{0.5em}

% Use the whole page
\usepackage{geometry}
% Extra math glyphs
\usepackage{amsmath}
% Proper enumerate spacing
\usepackage{enumitem}
% More pleasing screen fonts
\usepackage{lmodern}
% Fancy headers
\usepackage{fancyhdr}
\usepackage{graphicx}
% Allows absolute positioning of images
\usepackage{float}
% \usepackage[section]{placeins}
% Set no separation
\setlist{noitemsep}
% Set margins to reasonable
\geometry{margin=2.5cm}
% Sets graphics path
\graphicspath{ {./images/} }
% Sets up fancy headers

\addto\captionsenglish{
\renewcommand{\listfigurename}{List of Images}
}


\usepackage{listings}
\usepackage{color}

\pagestyle{plain}

\begin{document}


\definecolor{dkgreen}{rgb}{0,0.6,0}
\definecolor{gray}{rgb}{0.5,0.5,0.5}
\definecolor{mauve}{rgb}{0.58,0,0.82}

\lstset{frame=tb,
  language=Python,
  aboveskip=3mm,
  belowskip=3mm,
  showstringspaces=false,
  columns=flexible,
  basicstyle={\small\ttfamily},
  numbers=none,
  numberstyle=\tiny\color{gray},
  keywordstyle=\color{blue},
  commentstyle=\color{dkgreen},
  stringstyle=\color{mauve},
  breaklines=true,
  breakatwhitespace=true,
  tabsize=3
}

\pagestyle{fancy}
\fancyhf{}
\lhead{Natalie Hong - Visualisation Report}
\rhead{COSC3000}

\begin{titlepage}
\newgeometry{left=7.5cm} %defines the geometry for the titlepage
\noindent
\color{black}
\makebox[0pt][l]{\rule{1.3\textwidth}{1pt}}

{\Huge {Natalie Hong (45309740)}}
\vskip\baselineskip
\noindent
{\huge{COSC300 Graphics Report}}

\vskip\baselineskip
{\large {Semester 1 - 2020}}
\end{titlepage}

\newpage
\tableofcontents

\listoffigures

\newpage


\section{Introduction}
In this document, you will find my findings and attempts at creating a completed game. The game I had chosen to make was that of the supplied mega racer code.

To create the graphics required for this project, I used Python 3.5 and OpenGL to create objects and renderables within the computer graphics scene. I also used GitHub to allow for version control between each implemented feature that I added.

The features that were made were:
\begin{itemize}
	\item{1.1 - Scaling the Terrain}
	\item{1.2 - Setting up the Camera}
	\item{1.3 - Orientating and Placing the Racer Model}
	\item{1.4 - Texturing the Terrain}
\end{itemize}

In the following documentation. You will find screenshots and findings around each feature as well as how I went about approaching each feature within the project.

\section*{1.1 - Scaling the Terrain}
Scaling terrain consisted of scaling the terrain according to the path image supplied.
\begin{lstlisting}    
	# copy pixel 4 channels
        imagePixel = self.imageData[offset:offset+4];
        # Normalize the red channel from [0,255] to [0.0, 1.0]
        red = float(imagePixel[0]) / 255.0;

        xyPos = vec2(i, j) * self.xyScale + xyOffset;
	# TODO 1.1: set the height
	zPos = 0
\end{lstlisting}

Initially, the code supplied consisted of zPos =0 and I was to change it. I thought long and hard about to change and looked at the hints supplied. In it, I was told to use self.heightScale and to somehow utilize it in order to get a consistent z position for each terrain created. I had a thought that multiplying the red value calculated would allow me to get a a consistent z position. This was because red represented a value from 0 to 255 and multiplying it with the height scale would allow for different Z positions to be generated.

\begin{lstlisting}    
	# TODO 1.1: set the height
	zPos = self.heightScale * red
\end{lstlisting}

\begin{figure}[!ht]
	\centerline{\includegraphics[scale=0.35]{/1.1.png}}
	\caption{Step Graph Representing The Amount of Matches Played in Each State (2019)}
	\label{fig:figure2}
\end{figure}


This seemed to work and 



\end{document}

















