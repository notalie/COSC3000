% Document class options:
% 11pt for 11 point font
% oneside for not weird margins between odd and even pages
% (omitting oneside gives a better margin setup for print
% but i'm guessing  you're handing this in)
% a4paper to A4 page size, because LaTeX by default is setup
% to use US Letter paper size, because Donald Knuth
% the article class, as the book class is designed for
% larger documents and you probably don't need chapter
% environments
\documentclass[11pt, oneside, a4paper]{article}
\usepackage[utf8]{inputenc}
\usepackage[english]{babel}

\usepackage{hyperref}
\hypersetup{
    colorlinks=true,
    linkcolor=blue,
    filecolor=magenta,      
    urlcolor=cyan,
}

\urlstyle{same}
% Set spacing (i set it to 1.2x)
\renewcommand{\baselinestretch}{1}
% Indentation (set this to zero for normal prose)
\setlength{\parindent}{0em}
% Line breaking (spacing between paragraphs)
\setlength{\parskip}{0.5em}

% Use the whole page
\usepackage{geometry}
% Extra math glyphs
\usepackage{amsmath}
% Proper enumerate spacing
\usepackage{enumitem}
% More pleasing screen fonts
\usepackage{lmodern}
% Fancy headers
\usepackage{fancyhdr}
\usepackage{graphicx}
% Allows absolute positioning of images
\usepackage{float}

% Set no separation
\setlist{noitemsep}
% Set margins to reasonable
\geometry{margin=2.5cm}
% Sets graphics path
\graphicspath{ {./images/} }
% Sets up fancy headers
\pagestyle{fancy}
\fancyhf{}
\lhead{Natalie Hong Graphics Project Proposal}
\rhead{COSC3000}

\begin{document}
\section*{Introduction}
Hi, my name is Natalie Hong and I am a third year Software Engineering Student. I have chosen to take this course as I enjoy playing games as a past time, so it shouldn't be a surprise that I take lots of interest in this section of the course.

\section*{What?}
The computer graphics that I will be producing will be that of building upon supplied code. This supplied code is that of a basic game and my project will document my improvement upon it to hopefully create a visually pleasing experience.

Possible techniques that could be used:
\begin{itemize}
    \item Scaling
    \item Shadow Mapping
    \item Raytracing
    \item Transforming
    \item Axis-rotation
\end{itemize}

Possible features that could be added:
\begin{itemize}
    \item Model Placement
    \item Shaded Models
    \item Camera Tracking
    \item Tracked Lighting
    \item Prop Creation
    \item Particle Systems
\end{itemize}

\section*{Why?}
The reason I am using the supplied code is because I enjoy playing and creating games from scratch. With the supplied code, I am given the opportunity to work solely on the graphics component of the game which is a change to what I am used to. Racing games are also interesting as the terrain is always changing at constant speed, which I find quite unique.

\section*{How?}
To create the graphics required for this project, I will use Python 3.5 and OpenGL to create objects and renderables within the computer graphics scene. I will also be using GitHub to allow for version control between each implemented feature in case I make any mistakes and need to roll back.


\section*{Conclusion}
I hope you enjoyed this short proposal. I hope you look forward to what I can produce in the end. 

\end{document}