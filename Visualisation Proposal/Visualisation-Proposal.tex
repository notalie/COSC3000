% Document class options:
% 11pt for 11 point font
% oneside for not weird margins between odd and even pages
% (omitting oneside gives a better margin setup for print
% but i'm guessing  you're handing this in)
% a4paper to A4 page size, because LaTeX by default is setup
% to use US Letter paper size, because Donald Knuth
% the article class, as the book class is designed for
% larger documents and you probably don't need chapter
% environments
\documentclass[11pt, oneside, a4paper]{article}

% Set spacing (i set it to 1.2x)
\renewcommand{\baselinestretch}{1}
% Indentation (set this to zero for normal prose)
\setlength{\parindent}{0em}
% Line breaking (spacing between paragraphs)
\setlength{\parskip}{0.5em}

% Use the whole page
\usepackage{geometry}
% Extra math glyphs
\usepackage{amsmath}
% Proper enumerate spacing
\usepackage{enumitem}
% More pleasing screen fonts
\usepackage{lmodern}
% Fancy headers
\usepackage{fancyhdr}
\usepackage{graphicx}
% Allows absolute positioning of images
\usepackage{float}

% Set no separation
\setlist{noitemsep}
% Set margins to reasonable
\geometry{margin=2.5cm}
% Sets graphics path
\graphicspath{ {./images/} }
% Sets up fancy headers
\pagestyle{fancy}
\fancyhf{}
\lhead{Natalie Hong Visualisation Project Proposal}
\rhead{COSC3000}

\begin{document}
\section{Introduction}
Hi, my name is Natalie Hong and I am `first' year Software engineering 
student. I was originally undertaking UX as an IT major and ultimately decided 
to pursue a more development based career. I like programming, photography, 
design and cyber security. 

Now enough about me, let's get into the game. (also sorry in advance my spacing died :/)

\subsection{What?}
Faster than Water was originally a game concept built with the idea of the simulating 
being a pirate in a traversable world map. The title itself is a parody of popular game "Faster than Light" 
as the combat of the game is very similar to it. The main basis of the game was to collect four gems obtained by 
defeating bosses through combat. However, to defeat these bosses, different blueprints for ships needed to be collected
through ship combat. Depending on the type of ship that was challenged in combat could increase or decrease a player's notoriety 
allowing for different entity interactions. Certain NPCs were also available as well as a vast environment to explore. 
This idea of collecting gems was implemented in the story and offered unique opening and closing animations to our game. 

\section{Individual Work}

\subsection{Description}
For the majority of sprints I was involved with, I was in charge of visually implementing designed assets onto 
the game using the overlay renderer class and sprite batch methods. I also was part of the Creative Direction Team 
who was in charge of all major decisions and design consistency. 

A run down of my sprints is as follows:
\begin{enumerate}
    \item Sounds + Animation and basic sprites - Worked on fully implementing the sound manager for the 
    \item User Interface + Creative Direction Team - 
    \item Storyline team + Creative Direction Team - 
    \item Creative Direction Team - 
\end{enumerate}

\subsection{Methods}

I was also able to implement logic for  

% Put the H option to ensure that figures are where you want them to be
\begin{figure}[H]
    \centering
    % I am using a minipage to have two figures next to each other. If you just want one figure in a row, just remove the minipage structures entirely and just have:
    % \includegraphics[width=0.45\textwidth]{main_menu}
    % \caption{Main menu}
    % Always use scale in terms of page size -- NEVER scale images in terms of the original size, or else you will get unexpected behaviour
    \begin{minipage}{0.5\textwidth}
        \centering
        % Note that the width is relative to the parent minipage
        \includegraphics[width=0.9\linewidth]{amaru.png}
        \caption{Main menu}
    \end{minipage}%
    \begin{minipage}{0.5\textwidth}
        \centering
        \includegraphics[width=0.9\linewidth]{ash.png}
        \caption{Map view}
    \end{minipage}
\end{figure}
    
\end{document}
