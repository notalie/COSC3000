% Document class options:
% 11pt for 11 point font
% oneside for not weird margins between odd and even pages
% (omitting oneside gives a better margin setup for print
% but i'm guessing  you're handing this in)
% a4paper to A4 page size, because LaTeX by default is setup
% to use US Letter paper size, because Donald Knuth
% the article class, as the book class is designed for
% larger documents and you probably don't need chapter
% environments
\documentclass[11pt, oneside, a4paper]{article}
\usepackage[utf8]{inputenc}
\usepackage[english]{babel}

\usepackage{hyperref}
\hypersetup{
    colorlinks=true,
    linkcolor=blue,
    filecolor=magenta,      
    urlcolor=cyan,
}

\urlstyle{same}
% Set spacing (i set it to 1.2x)
\renewcommand{\baselinestretch}{1}
% Indentation (set this to zero for normal prose)
\setlength{\parindent}{0em}
% Line breaking (spacing between paragraphs)
\setlength{\parskip}{0.5em}

% Use the whole page
\usepackage{geometry}
% Extra math glyphs
\usepackage{amsmath}
% Proper enumerate spacing
\usepackage{enumitem}
% More pleasing screen fonts
\usepackage{lmodern}
% Fancy headers
\usepackage{fancyhdr}
\usepackage{graphicx}
% Allows absolute positioning of images
\usepackage{float}

% Set no separation
\setlist{noitemsep}
% Set margins to reasonable
\geometry{margin=2.5cm}
% Sets graphics path
\graphicspath{ {./images/} }
% Sets up fancy headers
\pagestyle{fancy}
\fancyhf{}
\lhead{Natalie Hong Visualization Project Proposal}
\rhead{COSC3000}

\begin{document}
\section*{Introduction}
Hi, my name is Natalie Hong and I am a third year Software Engineering Student. I have chosen to take this course as I enjoy visualising data spread and am super interested in the graphics section of the course as I was able to brush over it in highschool, but was never able to further my knowledge in it.
I also am heavily involved in a certain gaming community which has serves as the main factor of choice for my visualization project decision.

\section*{What?}
What I will be visualising will be Australian character data and usage throughout 2019 for the game Super Smash Bros. Ultimate for the Nintendo Switch. 
Since I am using an API and the data is always updating, I will be sorting each set of data into states and then into individual quarters of 2019.
Data will be visualised on a webpage as I would like it to be publically viewed when the project is completed.

The data will be presented as a bubble chart in which the positioning of the bubbles will be determined by elo (skill points) and total games played, with each bubble representing a character in the game. The radius of each bubble will represent the number of individual players of each character.
Since there are 70+ characters in this game, data visualization will be limited to displaying the top 5 characters.

\section*{Why?}
There are many reasons as to why I would like to undertake this project, so I'll summarise it for you:
\begin{itemize}
    \item I am curious as to if the amount of individual players will influence elo gain and/or matches played 
    \item I have a passion for this game and the data that the API offers
    \item During the holidays, I worked on many projects similar to this one utilising the same API so I have a sense of familiarity doing this project
    \item I enjoy web design and feel like this style of visualization will be visually pleasing 
\end{itemize}

\section*{How?}
To achieve the visualization page that I want I will use the following tools:
\begin{itemize}
    \item Python 3
    \item Bubble Chart JS library (to be decided)
    \item Ausmash API \href{https://api.ausmash.com.au/swagger/ui/index#/}{(Click for Documentation)}
    \item JQuery for Interaction display
\end{itemize}

\section*{Example Visualization}
Before diving into the code, I decided to make some very basic designs for what the final product could be. These designs were as follows:


\end{document}
